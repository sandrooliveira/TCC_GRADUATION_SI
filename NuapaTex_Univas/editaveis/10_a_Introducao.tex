%\chapter*{Introdução}
\begin{center}
  \vspace{1.2em}
  \textbf{\large INTRODUÇÃO}
  \vspace{2.9em}
\end{center}
\thispagestyle{empty}

\addcontentsline{toc}{chapter}{INTRODUÇÃO}
\stepcounter{chapter} %incrementa o número do capítulo

%\par Exemplo simples de parágrafo utilizando o comando\texttt{$\backslash$par}.
% Pode-se utilizar (in\-cons\-titucional (exemplo de forçar separação de sílabas)).

\par Sabe-se   que   atualmente   existem   diversos   softwares  
disponíveis no mercado.   Programas que   vão   desde   aqueles   desenvolvidos   sob   medida
até   softwares   genéricos   popularmente chamados   de   “softwares   de   prateleira”.
Porém   no   cenário   corporativo,   levando   em   consideração que   empresas   buscam
a   cada   dia   se   tornarem   mais   competitivas,   é   necessário   que   um   sistema  
ofereça   suporte   para   que   estas   possam   se tornar   mais   eficientes,
como   por exemplo,   auxiliar   na  busca   pela   diminuição   dos   custos   operacionais   para   que
assim   seja   possível   alcançar   os   melhores  preços de venda. \cite{livro_laudon_sistemas_de_informacoes_gerenciais} 

\par Neste   contexto,   a   ideia   de   agregar   características  
semelhantes  à inteligência   humana   aos programas se torna uma alternativa
interessante, pois segundo
\citeonline[p.329]{livro_laudon_sistemas_de_informacoes_gerenciais},

\begin{citacao}
Técnicas inteligentes ajudam os tomadores de decisão capturando o
conhecimento coletivo e individual, descobrindo padrões e  comportamentos
em grande quantidade de dados e gerando   soluções para problemas  amplos
e   complexos   demais   para   serem resolvidos por seres humanos.
\end{citacao} 

\par O conceito por trás deste pensamento é denominado Inteligência Artificial -
IA\footnotemark[1] - e é definido por \citeonline{artificial_intelligence}, como
uma área da ciência da computação que abrange a automatização da inteligência.

\par Para \citeonline[p.44]{ags_java_magazine}, “a IA é inspirada  em  
processos naturais e está relacionada à reprodução de   capacidades  
normalmente   associadas   à   inteligência   humana,   como aprendizagem,
adaptação,   o   raciocínio,   entre   outras”.   Ainda   segundo   os   mesmos
autores,   várias  abordagens   surgiram   ao   longo   da   história   tais   como  
a abordagem   conexionista,   inspirada   nos  neurônios   biológicos,   a  
simbolista, baseada   na   inferência   humana   e   a   evolutiva, fundamentada na teoria
de evolução das espécies.

\par Na busca por otimizar seu processo de produção, uma empresa da região, que
fabrica calças e aloca costureiras que trabalham em suas casas, deseja saber
qual a melhor forma de distribuir o trabalho para que um determinado lote de seu produto
seja produzido no menor tempo possível e com o menor custo. Para a resolução
deste problema, pode se fazer uso de IA através de um dos ramos da abordagem
evolutiva denominado Algoritmos Genéticos - AGs\footnotemark[2] - pois, como
afirma \citeonline{nocoes_geriais_anita_fernandes}, os AGs resolvem problemas de
otimização através de um processo que oferece como saída a melhor solução
dentro de várias possíveis formas de se resolver um problema.

\footnotetext[1]{O termo Inteligência Artificial será referenciado pela sigla IA
a partir deste ponto do trabalho.}

\footnotetext[2]{O termo Algoritmos Genéticos será referenciado pela sigla AGs
a partir deste ponto do trabalho.}

\par Para \citeonline {livro_ags_ricardo_linden}, AGs é definido
como uma técnica de otimização e busca que se baseia na teoria do processo de evolução e seleção
natural, proposto por Charles Darwin em seu livro A Origem das Espécies, que
afirma que indivíduos com melhor capacidade de adaptação ao seu ambiente possuem
maior chance de sobreviver e gerar descendentes.

\par Segundo \citeonline {nocoes_geriais_anita_fernandes}, o termo foi proposto
por Holland, em 1975, e por imitação à teoria da evolução, é representado por
uma população de indivíduos que representam soluções para um determinado problema e então tais soluções
podem evoluir até se chegar a uma solução ótima.

\par Vários trabalhos foram desenvolvidos utilizando a robustez de AGs, dentre
eles o trabalho de \citeonline{santos2007seleccao}, que faz a seleção de
atributos usando AGs para classificação de
regiões, o trabalho de \citeonline{silva2001otimizaccao}, que descreve a
otimização de estruturas de concreto armado utilizando AGs e 
o trabalho de \citeonline{freitas2007ferramenta} que descreve uma ferramenta
baseada em AGs para a geração de tabela de horário escolar.

\par Este trabalho irá descrever a criação de uma aplicação inteligente em
plataforma WEB utilizando AGs, para resolver o problema da empresa citada acima,
montando um agendamento que indique quais peças devem ser transportadas em quais
horários para quais empregadas, buscando encontrar a solução mais otimizada para
o processo.

%\par O \LaTeX~faz a ifenização automática, porém existem casos que é necessário
% forçá-lo. Veja no parágrafo anterior como forçar a ifenização, na palavra: inconstitucional.

%\par Existem várias formas de fazer referências. As duas formas mais comuns
% são: a primeira é assim: \cite{livro_laudon_sistemas_de_informacoes_gerenciais}, e a outra é mostrada conforme \cite{ecocentro}.


%