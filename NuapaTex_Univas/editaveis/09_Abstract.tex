% --- resumo em inglês ---

\begin{OnehalfSpacing} 

\noindent \imprimirAutorCitacaoMaiuscula. {\bfseries\imprimirtitulo}. {\imprimirdata}.  Monografia -- Curso de {\MakeUppercase\imprimircurso}, {\imprimirinstituicao}, {\imprimirlocal}, {\imprimirdata}.

\vspace{\onelineskip}
\vspace{\onelineskip}
\vspace{\onelineskip}
\vspace{\onelineskip}

\begin{resumo}[Abstract]%
\begin{otherlanguage*}{english}%
\textit{
\noindent The Genetic algorithms, inspired in genetic and in the Charles Darwin's theory of evolution of species,
are probabilist algorithms which is used to, randomly, make a parallel search for the best solution for a given problem. 
It is implemented based on principle of survival of the fittest, on the reproduction and on mutation of individuals and 
so it makes possible the evolution of the solutions which can make them even better. This research presents a general view 
of genetic algorithms concept, it demonstrate its fundamentals, operators and configurations and also makes a comparison 
with similar concepts in nature. To illustrate how this technique works it was developed an WEB based application with a 
genetic algorithm which was developed to find optimized solutions for tasks distribution in a production system of pants in 
which the employees work in their own houses and each part of the pants is produced separately. The software receives as 
input the time and the price of production per part of each employee and, after execution of the genetic algorithm over 
this data, it shows one good form to distribute the tasks in order to make the production in the shortest time and in the 
shortest cost within the planned deadline. This kind of research is applied because it will not modify the production processes, 
it will just optimize them.
}

\vspace{\onelineskip}
\vspace*{\fill}
\noindent \textbf{Key words}: \imprimirKeyWordOne. \imprimirKeyWordTwo. \imprimirKeyWordThree.
\end{otherlanguage*}
\vspace{\onelineskip}
\end{resumo}

\end{OnehalfSpacing}
