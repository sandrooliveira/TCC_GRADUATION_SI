
\chapter{CONCLUSÃO} 

\par Nos tempos atuais os sistemas de informação ocupam um papel crítico no cenário corporativo, pois
atuam como uma importante ferramenta no auxilio ao negócio. Softwares de gestão possibilitam
o acesso rápido e confiável à informações importantes, sendo estas características cruciais, uma vez que a 
informação tornou-se um dos patrimônios mais importantes das empresas. 

\par Além destas características, atualmente os softwares são dotados de inteligência artificial e vem cada vez 
mais realizando tarefas mais complexas oferendo um suporte primordial no apoio à tomadas de decisões e otimização 
de processos. Este trabalho visou explorar uma das abordagens da inteligência artificial em softwares, denominada 
algoritmos genéticos, produzindo uma aplicação capaz de otimizar a distribuição de atividades em uma linha de produção 
de calças. A escolha de explorar tal assunto se justifica pelo fato de que empresas buscam a cada dia se tornarem mais 
competitivas, neste contexto, a otimização de processos pode ser um dos fatores chaves para o aumento da competitividade.

\par Para demonstrar o conceito de algoritmos genéticos, foi desenvolvida, por meio deste trabalho, uma aplicação 
\texttt{web} que permite que o usuário desenhe seu processo, adicionando atividades a este em uma ordem de precedência,
além disso o usuário pode definir as costureiras de cada atividade juntamente com o tempo e o preço de produção de cada uma, assim 
com base nestas informações junto com a quantidade de peças e o tempo de produção, o algoritmo então distribui as atividades de 
forma a encontrar o menor tempo aliado ao menor custo de produção dentro do prazo de entrega. 

\par Para a construção do algoritmo genético, foi utilizado uma base desenvolvida pelo professor Artur Barbosa 
durante as aulas de sistemas especialistas, que define uma série de regras a ser seguida durante o desenvolvimento. 
Tal base é explicada com mais detalhes no quadro metodológico e foi de grande ajuda pois, além de definir as
regras, a base já implementava os métodos de seleção, cruzamento e mutação, neste sentido só foi preciso realizar algumas 
adaptações nestes para que pudessem se adequar à lógica desenvolvida para resolver o problema.

\par Um dos maiores desafios do trabalho foi a definição da função de avaliação, pois, 
uma vez que existe uma estrutura de nós contendo as atividades e o cálculo do tempo de 
cada atividade depende de nós predecessores, a maior parte da função foi pensada para ser 
desenvolvida de forma recursiva o que dificultou o \textit{debug}.

\par Assim, todos os objetivos propostos foram alcançados através dos conceitos e tecnologias descritos no quadro teórico e
dos procedimentos descritos no quadro metodológico. O software desenvolvido atendeu todos os requisitos do escopo definidos 
na seção de Reuniões do Quadro Metodológico. Porém, além do escopo definido, outras funcionalidades poderiam ser desenvolvidas, 
porém devido ao tempo disponível para o desenvolvimento do trabalho, não foi possível cobri-las. Uma delas é adição do custo de 
transporte das peças e materiais entre as costureiras, por enquanto só está sendo considerado o custo de produção por costureira. 
Um outro exemplo seria implementar uma restrição de forma que o algoritmo só distribuísse atividades para as costureiras disponíveis 
e além disso  implementar uma lógica de agendamento indicando quais peças devem ser transportadas em quais horários para quais 
costureiras.

\par No âmbito acadêmico, o presente trabalho, agregará à base de conhecimentos da Univás uma pesquisa com conceitos e exemplo de
implementação de um algoritmo genético, podendo ser utilizado como base para futuros trabalhos.

\par Conclui-se então que este trabalho foi de grande relevância, pois resolveu um problema complexo de ser solucionado em um 
tempo de processamento aceitável, tal problema poderia levar um tempo muito grande de processamento utilizando algoritmos
convencionais, além disto o trabalho proporcionou um sólido conhecimento sobre algoritmos genéticos aos pesquisadores.



