% --- resumo em português ---

\begin{OnehalfSpacing} 

\noindent \imprimirAutorCitacaoMaiuscula. {\bfseries\imprimirtitulo}. {\imprimirdata}.  Monografia -- Curso de {\MakeUppercase\imprimircurso}, {\imprimirinstituicao}, {\imprimirlocal}, {\imprimirdata}.

\vspace{\onelineskip}
\vspace{\onelineskip}
\vspace{\onelineskip}
\vspace{\onelineskip}

\begin{resumo}
~\\
%início do texto do resumo
\noindent Os algoritmos genéticos, inspirados na genética e na teoria da evolução das espécies de Charles Darwin,
são algoritmos probabilísticos que, aleatoriamente, promovem uma busca paralela pela melhor solução de um problema e são 
desenvolvidos baseados no princípio de sobrevivência dos mais aptos, na reprodução e na mutação dos indivíduos, de forma a fazer com 
que soluções evoluam e se tornem cada vez melhores. A presente pesquisa apresenta uma visão geral sobre
o conceito de algoritmos genéticos, demonstrando seus fundamentos, operadores e suas configurações, realizando um comparativo 
com os conceitos semelhantes na natureza. Para ilustrar a utilização desta técnica, foi desenvolvida uma aplicação, 
em plataforma WEB, juntamente com um algoritmo genético, que visa encontrar soluções otimizadas para distribuição 
das atividades de um sistema de produção de calças de modo que permita aos empregados trabalharem em suas 
casas e que cada parte da calça seja confeccionada separadamente. O software recebe, como entrada, dados de tempo e preço
de produção por peça de cada empregado e, após a execução do algoritmo genético sobre estes dados, apresenta como saída uma 
boa forma de distribuição das atividades que permita que a produção seja realizada com o menor tempo e tenha o menor custo possível 
dentro do prazo de entrega. Esta pesquisa é do tipo aplicada, pois não visa modificar os processos de produção e sim apresentar 
formas de otimizá-los.

%fim do texto do resumo
\vspace{\onelineskip}
\vspace*{\fill}
\noindent \textbf{Palavras-chave}: \imprimirPalavraChaveUm. \imprimirPalavraChaveDois. \imprimirPalavraChaveTres. \imprimirPalavraChaveQuatro.
\vspace{\onelineskip}
\end{resumo}

\end{OnehalfSpacing}
