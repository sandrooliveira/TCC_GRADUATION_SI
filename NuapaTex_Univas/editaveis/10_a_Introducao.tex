%\chapter*{Introdução}
\begin{center}
  \vspace{1.2em}
  \textbf{\large INTRODUÇÃO}
  \vspace{2.9em}
\end{center}
\thispagestyle{empty}


\addcontentsline{toc}{chapter}{INTRODUÇÃO}
\stepcounter{chapter} 

\par Sabe-se   que   atualmente   existem   diversos   softwares  
disponíveis no mercado para facilitar o dia a dia nas empresas.   
Programas que vão desde aqueles desenvolvidos sob   medida até   
softwares   genéricos   popularmente chamados   de   “softwares   
de   prateleira”. Porém, no cenário corporativo, levando em consideração 
que empresas buscam constantemente se tornarem  mais competitivas,   
é necessário que um sistema ofereça suporte para que essas possam se tornar   
mais eficientes, como por exemplo, auxiliar na busca pela diminuição dos custos   
operacionais, para que assim seja possível alcançar os melhores preços de venda
\cite{livro_laudon_sistemas_de_informacoes_gerenciais}.

\par Neste   contexto,   a   ideia   de   agregar   características  
semelhantes  à inteligência   humana   aos programas se torna uma alternativa
interessante, pois segundo
\citeonline[p.329]{livro_laudon_sistemas_de_informacoes_gerenciais},

\begin{citacao}
Técnicas inteligentes ajudam os tomadores de decisão capturando o
conhecimento coletivo e individual, descobrindo padrões e  comportamentos
em grande quantidade de dados e gerando   soluções para problemas  amplos
e   complexos   demais   para   serem resolvidos por seres humanos.
\end{citacao} 

\par O conceito por trás deste pensamento é denominado Inteligência Artificial -
IA\footnotemark[1] - e é definido por \citeonline{artificial_intelligence} como
uma área da ciência da computação que abrange a automatização da inteligência.

\par Para \citeonline[p.44]{ags_java_magazine}, “a IA é inspirada  em  
processos naturais e está relacionada à reprodução de   capacidades  
normalmente   associadas   à   inteligência   humana,   como aprendizagem,
adaptação,  raciocínio,   entre   outras”.   Ainda   segundo   os   mesmos
autores,   várias  abordagens   surgiram   ao   longo   da   história,   tais   como  
a abordagem   conexionista,   inspirada   nos  neurônios   biológicos;   a  
simbolista, baseada   na   inferência   humana   e   a   evolutiva, fundamentada na teoria
de evolução das espécies.

\par Na busca por otimizar seu processo de produção, uma empresa da região, que
fabrica calças e aloca costureiras que trabalham em suas casas, deseja saber
qual a melhor forma de distribuir o trabalho para que um determinado lote de seu produto
seja produzido dentro de um prazo com o menor custo possível. Para a resolução
desse problema, pode-se fazer uso de IA através de um dos ramos da abordagem
evolutiva denominado Algoritmos Genéticos - AGs\footnotemark[2] - pois, como
afirma \citeonline{nocoes_geriais_anita_fernandes}, os AGs resolvem problemas de
otimização através de um processo que oferece como saída a melhor solução
dentro de várias possíveis formas de se resolver um problema.

\footnotetext[1]{O termo Inteligência Artificial será referenciado pela sigla IA
a partir deste ponto do trabalho.}

\footnotetext[2]{O termo Algoritmos Genéticos será referenciado pela sigla AGs
a partir deste ponto do trabalho.}

\par O presente trabalho tem por objetivo geral desenvolver uma aplicação
utilizando técnicas de inteligência artificial capaz de realizar a alocação
de empregados em uma linha de produção de forma otimizada. 
\par Para a realização dessa pesquisa, foram colocados os seguintes objetivos
específicos: a) Demonstrar o uso de algoritmos genéticos; b) Projetar uma aplicação em plataforma WEB que distribua as
atividades de uma linha de produção de calças de forma inteligente, a fim de se
obter um tempo de produção dentro de um prazo estipulado, procurando obter o menor custo.

\par Para \citeonline {livro_ags_ricardo_linden}, AGs é definido
como uma técnica de otimização e busca que se baseia na teoria do processo de evolução e seleção
natural, proposto por Charles Darwin em seu livro A Origem das Espécies, que
afirma que indivíduos com melhor capacidade de adaptação ao seu ambiente possuem
maior chance de sobreviver e gerar descendentes.

\par Segundo \citeonline {nocoes_geriais_anita_fernandes}, o termo foi proposto
por Holland, em 1975, e por imitação à teoria da evolução, é representado por
uma população de indivíduos que representam soluções para um determinado problema e então tais soluções
podem evoluir até se chegar a uma solução ótima.

\par Vários trabalhos foram desenvolvidos utilizando a robustez de AGs, entre
eles o trabalho de \citeonline{santos2007seleccao}, que faz a seleção de
atributos usando AGs para classificação de
regiões; o trabalho de \citeonline{silva2001otimizaccao}, que descreve a
otimização de estruturas de concreto armado utilizando AGs e 
o trabalho de \citeonline{freitas2007ferramenta}, que descreve uma ferramenta
baseada em AGs para a geração de tabela de horário escolar.

\par O sistema desenvolvido neste trabalho, assim como nos outros citados, 
tem o mesmo conceito, o de otimizar processos. No caso da fábrica de calças, 
tal otimização irá promover a diminuição do custo de produção das calças confeccionadas, 
permitindo então que essas possam ser vendidas também por um preço melhor, 
beneficiando assim, os consumidores da região. 
Além disso, a ideia de otimizar procedimentos, muitas vezes, também contribui
para a preservação de recursos naturais devido ao fato de que processos otimizados 
podem significar economia de energia e diminuição de emissão de gases. 
No caso da empresa de calças, se o gestor tiver sempre em mãos a solução mais otimizada, 
terá consequentemente um menor tempo de transporte de materiais, o que reduzirá a emissão de gases 
dos veículos utilizados no transporte de materiais e peças entre as costureiras.

\par No âmbito acadêmico, o trabalho agregará à base de conhecimento
da Univás um material que faça referência a tecnologias e conceitos de inteligência
artificial, hoje presentes em diversos sistemas críticos de apoio à
decisão e otimização de processos nas empresas.
