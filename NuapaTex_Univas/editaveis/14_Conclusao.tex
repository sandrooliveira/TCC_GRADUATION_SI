
\chapter{CONCLUSÃO} 

\par Nos tempos atuais, os sistemas de informação ocupam um papel crítico no cenário corporativo, pois
atuam como uma importante ferramenta no auxílio ao negócio. Softwares de gestão possibilitam
o acesso rápido e confiável a informações importantes. Tais características são cruciais em um sistema, uma vez que a 
informação tornou-se um dos patrimônios mais importantes das empresas. 

\par Além destas características, os softwares são dotados de inteligência artificial e vêm cada vez 
mais realizando tarefas mais complexas, oferendo um suporte primordial no apoio a tomada de decisões e otimização 
de processos. Este trabalho visou explorar uma das abordagens da inteligência artificial em softwares, denominada 
algoritmos genéticos, produzindo uma aplicação capaz de otimizar a distribuição de atividades em uma linha de produção 
de calças. A escolha de explorar tal assunto se justifica pelo fato de que empresas buscam constantemente se tornar mais 
competitivas, nesse contexto, a otimização de processos pode ser um dos fatores chave para o aumento da competitividade.

\par Para demonstrar o conceito de algoritmos genéticos, foi desenvolvida, por meio deste trabalho, uma aplicação 
\textit{web} que permite que o usuário modele seu processo, adicionando atividades a ele em uma ordem de precedência,
além disso o usuário pode definir as costureiras de cada atividade juntamente com o tempo e o preço de produção de cada uma. Assim, 
com base nessas informações, junto com a quantidade de peças e o tempo de produção, o algoritmo então distribui as atividades de 
forma a encontrar o menor tempo, aliado ao menor custo de produção dentro do prazo de entrega. 

\par Para a construção do algoritmo genético, foi utilizada uma base desenvolvida pelo professor Artur Barbosa, 
durante as aulas de sistemas especialistas, que define uma série de regras a serem seguidas durante o desenvolvimento. 
Tal base é explicada com mais detalhes no quadro metodológico e foi de grande ajuda, pois além de definir as
regras, a base já implementava os métodos de seleção, cruzamento e mutação, sendo preciso apenas realizar algumas 
adaptações neles para que pudessem se adequar à lógica desenvolvida para resolver o problema.

\par Um dos maiores desafios do trabalho foi a definição da função de avaliação, pois, 
uma vez que existe uma estrutura de nós contendo as atividades, e o cálculo do tempo de 
cada atividade depende de nós predecessores, a maior parte da função foi  
desenvolvida de forma recursiva o que dificultou o \textit{debug}.

\par Assim, todos os objetivos propostos foram alcançados através dos conceitos e tecnologias descritos no quadro teórico e
nos procedimentos descritos no quadro metodológico. O software desenvolvido atendeu todos os requisitos do escopo definidos 
na seção~\ref{reunioes} do Quadro Metodológico. Porém, além do escopo definido, outras funcionalidades poderiam ser desenvolvidas, 
mas devido ao tempo disponível para o desenvolvimento do trabalho, não foi possível cobri-las. Uma delas é adição do custo de 
transporte das peças e materiais entre as costureiras, por enquanto está sendo considerado somente o custo de produção por costureira. 
Um outro exemplo seria implementar uma restrição de forma que o algoritmo só distribuísse atividades para as costureiras disponíveis. 
Além disso  implementar uma lógica de agendamento indicando quais peças devem ser transportadas em quais horários para quais 
costureiras.

\par No âmbito acadêmico, o presente trabalho, agregará à base de conhecimentos da Univás uma pesquisa com conceitos e exemplo de
implementação de um algoritmo genético, podendo ser utilizado como base para futuros trabalhos.

\par Conclui-se então que este trabalho foi de grande relevância, pois resolveu um problema complexo de ser solucionado em um 
tempo de processamento aceitável, tal problema poderia levar um tempo muito grande de processamento utilizando algoritmos
convencionais, além disso o trabalho proporcionou um sólido conhecimento sobre algoritmos genéticos aos pesquisadores.



