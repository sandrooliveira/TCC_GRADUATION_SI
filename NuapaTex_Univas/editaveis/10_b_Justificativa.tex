\chapter{JUSTIFICATIVA}

\par A otimização de processos nas empresas promove a diminuição
de custos permitindo  com que estas possam vender seus produtos por um preço
melhor. Além disso, a ideia de otimizar procedimentos, muitas vezes, também
contribui para a preservação de recursos naturais devido ao fato de que processos otimizados
podem significar economia de energia e diminuição de emissão de gases. No
caso da empresa de calças, se o gestor ter sempre em mãos a solução mais
otimizada, irá ter um menor tempo de transporte de materiais, o que irá reduzir a emissão de gases
de seus veículos na natureza. Além disso, ele poderá vender as calças por um preço
melhor o que irá beneficiar os consumidores da região.

\par Isso justifica a criação de uma aplicação inteligente que, através de uma
gama de soluções para o planejamento da produção, encontre aquela que seja a
mais eficiente.

\par No âmbito acadêmico, o trabalho agregará à base de conhecimento da Univás
um material que faça referência a tecnologias e conceitos de inteligência
artificial, que hoje estão presentes em diversos sistemas críticos de apoio a
decisão e otimização de processos nas empresas.
